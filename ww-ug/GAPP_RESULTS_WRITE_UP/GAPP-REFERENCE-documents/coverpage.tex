%% LyX 2.1.1 created this file.  For more info, see http://www.lyx.org/.
%% Do not edit unless you really know what you are doing.
\documentclass[english]{article}
\usepackage[T1]{fontenc}
\usepackage[latin9]{inputenc}
\usepackage{babel}
\begin{document}

\title{Consumption Bundle Aggregation in Poverty Measurement: Implications
for Poverty and its Dynamics in Uganda}


\author{Bjorn Van Campenhout%
\thanks{International Food Policy Research Institute (IFPRI) - Kampala, b.vancampenhout@cgiar.org%
}, Haruna Sekabira%
\thanks{Department for Agricultural Economics and Rural Development, Goettingen
University%
}, and Dede H. Aduayom%
\thanks{International Food Policy Research Institute (IFPRI) - Kampala%
}}
\maketitle
\begin{abstract}
Official poverty figures in Uganda are flawed by the fact that the
underlying poverty lines are based on a single national food basket
that was constructed in the early 1990s. In this paper, we estimate
a new set of poverty lines that accounts for the widely divergent
diets throughout the country using the latest available household
survey. Using these updated poverty lines, we then look at poverty
dynamics using four waves of the Uganda National Panel Survey. We
classify households into categories depending on their change in poverty
status over time and relate this to characteristics that are likely
to change only slowly. This enables us to explore the characteristics
of households that, for instance, grow out of poverty and how they
differ from households that appear to be trapped in poverty. Our approach
generates poverty measures that are more credible from a theoretical
point of view and are more in line with what other researchers find.\end{abstract}
\begin{description}
\item [{JEL:}] O12, D31, O55
\item [{Keywords:}] poverty, cost of basic needs, revealed preferences,
Uganda
\item [{note:}] This project is under revision control. All source code
to replicate the analysis can be found on  : https://bitbucket.org/bjvca/wider/\end{description}

\end{document}
