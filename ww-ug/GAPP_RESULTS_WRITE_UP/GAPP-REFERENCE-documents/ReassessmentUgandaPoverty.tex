%% LyX 2.0.6 created this file.  For more info, see http://www.lyx.org/.
%% Do not edit unless you really know what you are doing.
\documentclass[12pt,english]{article}
\usepackage[T1]{fontenc}
\usepackage{amsthm}
\usepackage{amsmath}
\usepackage[authoryear]{natbib}

\makeatletter

%%%%%%%%%%%%%%%%%%%%%%%%%%%%%% LyX specific LaTeX commands.
%% Because html converters don't know tabularnewline
\providecommand{\tabularnewline}{\\}

\makeatother

\usepackage{babel}
\begin{document}

\title{The Role of Price and Consumption Bundle Aggregation in Poverty Measurement:
A Reassessment of Poverty in Uganda}


\author{Bjorn Van Campenhout, Haruna Sekabira and Dede Houeto Aduayom}
\maketitle
\begin{abstract}
Uganda has seen substantial poverty reduction, but reported measures
are flawed because the official poverty line is based on a single
national food poverty line that was constructed using 1993 data. We
use regional consumption patterns to construct poverty lines that
account for location specific diets and adjust these poverty lines
such that they pass revealed preference tests. In doing so, we estimate
poverty in Uganda using the 2005/06 and 2009/10 Uganda National Household
Survey and the 2010/11 round of the Uganda National Panel Survey.
We find poverty to be higher in general, and reductions in poverty
were slower than official figures suggest. We also find that poverty
is highest in regions that consume relatively more matooke, while
the North seems to enjoy cheap calories, making them less poor than
officially reported. However, the figure from the North hides significant
heterogeneity, and poverty reduction in the North is very slow. 
\end{abstract}

\section{Introduction}

During the past few decades, Uganda has seen substantial economic
growth. Since 1986, when the National Resistance Movement (NRM) took
over government, real GDP has grown at an annual rate of 6.8, making
it one of the fastest growing economies in Africa. This growth has
been attributed to the new government that has implemented a far-reaching
economic reform agenda, transforming Uganda into one of the most liberal
economies in Sub-Saharan Africa. With the liberalization of the exchange
and trade regime, the abolition of the Industrial Licensing Act, the
promulgation of a new investment code, and the gradual liberalization
of agricultural pricing and marketing, the government has put in place
some of the essential pre-conditions for sustainable growth. 

This growth has been accompanied by equally impressive declines in
the levels of poverty as reported by the government. While aggregate
headcount poverty stood at about 57 percent in the early 1990s, the
most recent estimate puts 24 percent of the population below the official
poverty line. But despite these successes at the aggregate level,
researchers warn that this growth has not been shared equally by the
population at large. Marked spatial heterogeneity in baseline poverty
and subsequent poverty reductions mean differences in the standard
of living between locations are much higher now than what they used
to be. In addition, some argue that the official poverty measures
are too high if they are compared to non-monetary indicators and qualitative
data from other countries. These observations have raised suspicion
about the poverty lines used by the government of Uganda.

In light of this, we will reassess poverty in Uganda paying particular
attention to spatial aspects in the construction of poverty lines.
Uganda has been using the same Cost of Basic Needs (CBN) basket for
over two decades now. In addition, this consumption basket is the
same for the entire Ugandan population. However, it is well known
that Uganda has an unusually large regional variation in diets within
the country, with no less than six major food staples being eaten
\citep{RePEc:oup:jafrec:v:12:y:2003:i:4:p:598-624}. For instance,
matooke, or cooking bananas, is mainly eaten in the West, and to a
lesser extent in Central region. Households living in the dryer North,
however, obtain a substantial share of their calories through sorghum
and sesame. These diets are not only determined by custom, they are
also related to the agro-geological characteristics. For instance,
the dryer North is more suited for cassava than matooke, and the low
value to weight ratio typical of staple foods essentially renders
them non-tradables. If these different diets result in different costs
to obtain basic needs, this should be accommodated in the poverty
measures.

More specifically, we will estimate poverty at three distinct points
in time during the last decennium in Uganda. To account for differences
in consumption patterns, we will construct regional poverty lines.
For each of the four regions, we construct a food basket that produces
a certain minimum of calories that reflects the diets of the poorest
households in that region. These baskets are then multiplied by prices
prevailing in that region to come to food poverty lines. An allowance
for basic non-food necessities is than added to get four regional
poverty lines. We then test these four poverty lines to see if they
are utility consistent. The idea is that a basic needs bundle in a
certain region A should always be cheaper than a bundle from any other
region valued at prices of region A. If a bundle does not satisfy
these revealed preference conditions, we use an information theoretic
approach to adjust the bundles until they do.

The remainder of this paper is structured as follows. We first give
an overview of poverty in the previous millennium and look at the
present official poverty estimates. In section 3, we briefly explain
the reasoning behind the use of regional poverty lines and the role
of revealed preferences when constructing them. Section 4 presents
our poverty estimates and compares them to poverty figures found in
other studies. Section 5 concludes. 


\section{Poverty in Uganda - Trends and Controversies}

There is a substantial literature on poverty and its dynamics in the
1990s. This is due to the fact that there is relatively high quality
data available for this period. Uganda has been conducting national
household surveys since 1992, and has conducted six such surveys before
the turn of the millennium. Some of these data sets form a panel.
The first data set, the Integrated Household Survey conducted in 1992-3
can be linked to the last survey of the millennium, referred to as
the first Uganda National Household Survey (UNHS 1999-0). In between
these two surveys, a series four Monitoring Surveys (1993-4 up to
1999-0) have been carried out. More information on these data sets
can be found in \citet{RePEc:taf:jdevst:v:42:y:2006:i:7:p:1225-1251}.

Table \ref{tab:Official-Poverty-Headcountsninties}, taken from \citet{RePEc:oup:jafrec:v:12:y:2003:i:4:p:598-624},
presents official headcount poverty rates before the year 2000. As
can be seen, poverty decreased substantially, falling by almost 40
percent at the national level. However, this table also shows significant
spatial differences in both levels and changes in poverty. The urban
areas and Central region reduce poverty fastest. The Northern region,
already starting from high levels of poverty, are relatively unsuccessful
in bringing down the number of people living below the poverty line.
In addition, studies that exploit the panel nature of the data find
that in some regions, poverty is particularly persistent \citep{DPR:DPR220}. 

\begin{table}
\caption{Official Poverty Headcounts 1992-2000\label{tab:Official-Poverty-Headcountsninties}}


\begin{centering}
\begin{tabular}{ccccccc}
\hline 
 & 1992-3 & 1993-4 & 1994-5 & 1995-6 & 1997-8 & 1999-0\tabularnewline
\hline 
\hline 
national & 55.7 & 51.2 & 50.2 & 49.1 & 44.4 & 35.1\tabularnewline
 &  &  &  &  &  & \tabularnewline
urban & 27.8 & 21.0 & 21.5 & 19.8 & 16.7 & 10.1\tabularnewline
rural & 59.7 & 55.6 & 54.3 & 53.7 & 48.7 & 39.0\tabularnewline
 &  &  &  &  &  & \tabularnewline
Central & 45.6 & 34.5 & 30.3 & 30.4 & 27.9 & 20.1\tabularnewline
Eastern & 58.5 & 57.6 & 65.3 & 58.4 & 54.3 & 37.3\tabularnewline
Northern & 72.2 & 69.3 & 63.5 & 70.2 & 59.8 & 64.8\tabularnewline
Western & 53.1 & 53.9 & 50.9 & 46.3 & 42.8 & 28.0\tabularnewline
\hline 
\end{tabular}
\par\end{centering}

\centering{}note: reproduced from \citet{RePEc:oup:jafrec:v:12:y:2003:i:4:p:598-624}.
\end{table}


One controversy which we will also address in this paper refers to
the fact that the official poverty estimates are based on poverty
lines that are rooted in a single consumption bundle. \citet{RePEc:oup:jafrec:v:12:y:2003:i:4:p:598-624}
and \citet{New1} argue that Uganda is unusual in that different regions
have very different diets. This may not matter very much if the diets
are equally cost effective in obtaining basic needs. However, the
staple food of choice of a large part of the population, both in the
West and the Central region, is matooke, a highly localized staple.
Matooke appears to be a very expensive source of calories, compared
to what people in for instance the North consume. When they account
for this in their analysis, they come to the conclusion that poverty
is much more pronounced in the Western region than in the North. Even
after correcting for income difference, as regions that consume more
expensive calories may do so simply because they have higher incomes,
\citet{RePEc:oup:jafrec:v:12:y:2003:i:4:p:598-624} come to the conclusion
that the Western region overtakes the North as the poorest region
using 1993 data. 

While the above refers to the nineties, progress in the first decade
of the new millennium is equally impressive. Table \ref{tab:Official-poverty-headcount00s}
shows that poverty at the national level kept falling at the same
rate. At the same time, differential progress in poverty reduction
in different regions leads to higher poverty differences between regions
by 2009/10. For instance, by 2009/10, poverty is more than four times
higher in the Northern region than in the Central region. In 2002/03,
the North was 2.7 times poorer than the Central Region.

\begin{table}


\caption{Official poverty headcounts 2002-2010\label{tab:Official-poverty-headcount00s}}


\centering{}%
\begin{tabular}{cccc}
\hline 
 & 2002/03 & 2005/06 & 2009/10\tabularnewline
\hline 
\hline 
national & 38.3 & 31.1 & 24.5\tabularnewline
 &  &  & \tabularnewline
urban & 14.4 & 13.7 & 9.1\tabularnewline
rural & 42.7 & 34.2 & 27.2\tabularnewline
 &  &  & \tabularnewline
Central & 22.3 & 16.4 & 10.7\tabularnewline
Eastern & 46.0 & 35.9 & 24.3\tabularnewline
Northern & 63.0 & 60.7 & 46.2\tabularnewline
Western & 32.9 & 20.5 & 21.8\tabularnewline
\hline 
\end{tabular}
\end{table}


\citet{New2} uses Demographic and Health Survey Data and methods
related to poverty mapping and small area estimation to look at poverty
trends across Uganda from 1995 to 2010. She uses the 2005/06 UNHS
survey to estimate regressions that correlate poverty to a series
of household characteristics that also appear in the DHS (four such
surveys have been carried out between 1995 and 2009/10). She then
uses the DHS surveys to predict poverty in each of the DHS survey
years. She finds that poverty indeed reduced over time, but much slower
than official figures suggest. While her national estimate of headcount
poverty in 2006 is 33\% and thus very close to the official estimate
of 2005/6, the rates still stands as 30 percent using the 2009 DHS,
about 6 percentage points higher then the 2009/10 UNHS estimate.

More in general, there is a view amongst publicists and opinion makers
in Uganda that the poverty figures reported by the government of Uganda
are too optimistic. \citet{New3} calls official reported poverty
changes ``a fiction''. \citet{New4} admits that qualitative findings
on poverty trends suggest there was a decrease in well-being despite
the drop in poverty rates. Recently, an unpublished manuscript has
been circulating that compares Uganda to other African countries on
six non-monetary poverty indicators, such as literacy rates and access
to piped water. This admittedly partial analysis also points to a
much higher incidence of poverty. 


\section{Utility Consistent Poverty Lines using Revealed Preferences}

One of the main weaknesses of the official poverty measures is the
fact that it is based on a poverty line that is constructed using
a single food commodity bundle for the entire country. In addition,
this food basket was constructed in 1993 and has not been updated
since, apart from simple inflation by the consumption price index.
However, it is well known that in many instances - for example, if
relative prices of basic commodities vary by region (or through time)
and preferences permit substitution - the use of a single consumption
bundle may yield inconsistent poverty comparisons \citep{RePEc:ucp:ecdecc:v:51:y:2002:i:1:p:77-108}.
While difference in prices in different locations have always been
incorporated by adjusting local prices to the prices used in the construction
of the poverty line, it becomes more and more common to also account
for spatial heterogeneity in consumption bundles (e.g. \citealp{RePEc:bla:revinw:v:52:y:2006:i:3:p:399-421},
\citealp{RePEc:eee:wdevel:v:31:y:2003:i:2:p:339-358}).

While differences in consumption baskets are interesting in its own
right, they only become relevant in the context of poverty measurement
and analysis when we evaluate the cost of these basic needs. Indeed,
what matters for poverty comparisons is that different diets may provide
the same basic needs (usually a given amount of kilo-calories per
day per adult equivalent) at significantly different cost. It is especially
in this regard that Uganda is an interesting case. Matooke, the main
ingredient in the diet of households in the West is a very expensive
source of food energy, almost three times as expensive as sorghum
that is consumed in the North. As such, to compare the West to the
North on the basis of the same daily nutritional requirements, the
West will need a much higher poverty line.

But how can we be sure that two different consumption bundles provide
the same basic needs? Or, in the language of \citet{RePEc:oup:wbecrv:v:8:y:1994:i:1:p:75-102},
how do we assure consistency%
\footnote{A poverty measure is consistent if two individuals at the same welfare
level are considered equally poor. %
}? CBN poverty lines can be viewed as the expenditure needed to acquire
a specific bundle of goods. Different CBN poverty lines will be utility
consistent if the underlying bundles of goods are on the Hicksian
utility-compensated demand functions and hence yield the same level
of utility. But this only passes the question down. How can we be
sure the underlying bundles yield the same utility? As \citet{RePEc:ucp:ecdecc:v:58:y:2010:i:3:p:449-474}
argue, the theory of revealed preferences provides a framework for
countering these difficulties. 

The idea uses the rationality assumption that economic agent that
derive utility from consumption always preferred consuming more to
less. Let us assume that a representative agent living in spatial
domain \emph{$(r\in R)$ }derives utility from a set of consumption
goods ($i\in I$). We will then instruct each representative consumer
in each spatial domain \emph{r }to spend a minimum to attain an arbitrary
(but constant across spatial domains) level of utility. As such, each
individual will spend $\sum p_{i,r}q_{i,r}$ on a consumption bundle,
with $p_{i,r}$ the price of good \emph{i} in spatial domain \emph{r}
and $q_{i,r}$ the quantity of good \emph{i} in spatial domain \emph{r}.
Revealed preference conditions will then imply that:

\begin{equation}
\sum p_{i,r}q_{ir'}\geq\sum p_{i,r}q_{i,r}\;\forall r,r'\label{eq:revpref}
\end{equation}


This is so because the representative consumer in spatial domain \emph{r}
will choose only that bundle that minimizes expenditure. Thus, any
other bundle that yields the same level of utility (such as for instance
the one chosen by the representative consumer in region \emph{r'})
should be equally or more expensive than the chosen bundle. There
can be no bundle that costs less than the chosen one yet yields that
same utility, because then the rational consumer should have chosen
that one. The above condition \eqref{eq:revpref} should hold for
all possible pairs of spatial domains.

In practice, however, it will be hard to construct a set of poverty
lines that meet revealed preference conditions for all possible pairs
of spatial domains. We use a minimum cross-entropy approach to adjust
expenditure shares such that they meet revealed preference conditions.
This approach uses the expenditure shares of the original bundles
as prior information (in the form of probabilities that an arbitrarily
small amount of money will be devoted to the purchase of the particular
good) and the revealed preference conditions as constraints on the
values that the parameters can take. The end result will be a set
of adjusted expenditure shares that are as closely as possible to
the original shares, yet that obey a minimal set of conditions such
that the estimated bundles are consistent with some arbitrary unknown
preference set \citep{RePEc:ucp:ecdecc:v:58:y:2010:i:3:p:449-474}. 


\section{A Reassessment of Poverty in Uganda}

We look at data taken from three different surveys. The oldest is
the Uganda National Household Survey conducted in 2005/06. The second
is the Uganda National Household survey conducted in 2009/10. Finally,
we also run the analysis on the second wave of the Uganda National
Panel Survey (2010/11). Unfortunately, this last survey is much smaller
than the previous surveys, so this survey is designed to be representative
only up to the regional level.

For each data set, we start by constructing new food bundles in each
spatial domain. For Uganda, we use the four regions as spatial domains.
In each poverty line region, a basket of food products that satisfied
basic calorie needs (WHO 1985)%
\footnote{This is how we fix the utility level.%
} was identified using information on the age and sex composition of
the household and the recorded consumption patterns of poorer households.
The cost of this basket, valued at prices prevailing within each region,
is the food poverty line in each region. A nonfood poverty line was
obtained for each region by calculating the share of food expenditures
for households whose total food and nonfood consumption per capita
was near the food poverty line. The total poverty line is obtained
as the sum of the food and the nonfood poverty lines. Finally, if
the bundles do not satisfy revealed preference tests, we adjust expenditure
shares until they do using the entropy approach outlined above. 

It will be instructive to have a closer look at the poverty lines.
After all, poverty lines are not only useful to separate the rich
from the poor,or but also as cost-of-living indexes, permitting interpersonal
welfare comparisons when the cost of acquiring basic needs varies
over time or space \citep{RePEc:fth:wobali:133}. Table \ref{tab:Evolution-of-cost}
shows that the cost to satisfy basic needs is highest in the Central
region and lowest in the Northern region. The fact that Northern region
has the lowest cost of basic needs is due to the fact that the diet
in this region consists mainly of sweet potatoes, cassava, sorghum
and sesame. These four commodities are cheap sources of calories.
Both sweet potatoes and cassava cost only UGX0.088 per kilo-calorie.
Sorghum costs UGX0.064 and sesame only UGX0.037 per kg. In the West,
and to a lesser extent in the Central region, sweet potatoes is eaten
together with matooke (UGX0.174 per kg) and cassava. In the East,
sweet potatoes and cassava is supplemented with maize (UGX0.128) and
millet (UGX0.143)%
\footnote{These figures are taken from table 4 in \citet{RePEc:oup:jafrec:v:12:y:2003:i:4:p:598-624}
and are median 1993-4 prices calculated from the first monitoring
survey. While these prices will certainly be higher in the 2005/06
to 2010/11 period due to inflation, we think the price differences
will be largely preserved.%
}.

\begin{table}
\caption{Evolution of cost of basic needs\label{tab:Evolution-of-cost}}


\centering{}%
\begin{tabular}{ccccccc}
 & \multicolumn{2}{c}{2005/06} & \multicolumn{2}{c}{2009/10} & \multicolumn{2}{c}{2010/11}\tabularnewline
\cline{2-7} 
 & poverty line & food share & poverty line & food share & poverty line & food share\tabularnewline
\hline 
Central & 827.73 & 43.89\% & 1387.87 & 73.16\% & 1111.45 & 75.61\%\tabularnewline
Eastern & 608.45 & 59.33\% & 788.26 & 68.69\% & 755.15 & 79.10\%\tabularnewline
Northern & 305.91 & 47.60\% & 609.38 & 70.75\% & 622.03 & 74.60\%\tabularnewline
Western & 746.75 & 56.21\% & 834.79 & 72.95\% & 967.00 & 80.09\%\tabularnewline
\hline 
\end{tabular}
\end{table}


We can also compare our poverty lines to the official poverty lines.
As said before, official poverty lines are based on a single food
basket, derived from the 1993/1994 data. In particular, the food basket
was identified with 28 of the most frequently consumed food items
by households with less than the median income. These food items were
then converted into their caloric equivalent and scaled to generate
3000 calories per adult equivalent per day using as reference the
WHO estimates for an 18-30 year old male. Next, a non-food allowance
is added. Non-food requirements are estimated as the average non-food
expenditure of those households whose total expenditure is just equal
to the food poverty line. The non-food allowance does allow for spatial
heterogeneity, as separate averages are calculated for urban/rural
location interacted with the four regions. This is a conservative
estimate of the poverty line, as one could also opt to estimate the
average non-food expenditure of those households with food expenditure
equal to the food poverty line. Official poverty lines for the 2005/06
and 2009/10 surveys are presented in table \ref{tab:Official-poverty-lines}.

\begin{table}
\caption{Official poverty lines\label{tab:Official-poverty-lines}}


\centering{}%
\begin{tabular}{ccccc}
 & \multicolumn{2}{c}{2005/06} & \multicolumn{2}{c}{2009/10}\tabularnewline
\cline{2-5} 
 & poverty line & food share & poverty line & food share\tabularnewline
\hline 
Central & 728.65 & 70.12\% & 1014.90 & 69.82\%\tabularnewline
Eastern & 692.24 & 73.81\% & 959.96 & 73.81\%\tabularnewline
Northern & 700.15 & 72.97\% & 969.02 & 73.12\%\tabularnewline
Western & 680.28 & 75.10\% & 943.21 & 75.12\%\tabularnewline
\hline 
\end{tabular}
\end{table}


Comparing the official poverty lines to the utility consistent poverty
lines underscores the importance of the cost of diets in different
regions to attain a certain amount of kilo-calories. The most striking
case is again the North. According to official estimates, it will
cost a person UGX700 to obtain his/her daily calorie requirements.
However, if we allow this person to have his/her own specific regional
diet, he/she will extract energy from staples that are potentially
more or less cost effective in obtaining basic needs. In the North,
this results in much higher shares of cassava, sorghum and sim-sim
(sesame), all crops that are relatively cheap in obtaining a given
amount of kilo-calories. In 2005/06, the cost of basic needs of an
individual for one day in the North appears to be less than half of
the official cost once we take into account this region specific diet. 

The reverse holds for the Central and the Western regions. There,
the cost of meeting basic needs of an individual increases if we allow
them to assemble their own diet. This is as expected, as these regions
rely heavily on matooke, the most expensive staple%
\footnote{Irish potatoes and rices are even less cost effective than matooke,
but these crops are rarely consumed in Uganda, especially in rural
Uganda.%
}. The official poverty line that relies on a single consumption basket
for the entire country therefore averages out this expensive diet
over the entire country. The Eastern region, where cassava and sweet
potatoes is supplemented with maize, the utility consistent poverty
line is slightly lower than the official line in 2005/06.

It is also interesting to look at the evolution over time. For the
official poverty line, the consumption bundle has simply been re-weighted
by the food prices that prevail in 2009/10. Then the non-food allowance
is re-estimated using the method presented above. This means that
the increase in the cost of living is largely the same in each region,
about UGX300. The only exception is Central, which is heavily urbanized
and included Kampala. The story is different when using revealed preferences.
The cost of living changes little in the West, reflecting the fact
that matooke is a very localized commodity whose price evolved somewhat
independent from the food prices of other staples that surged during
2008/09. The Central region saw a more than UGX500 increase in the
cost of living. This is partly due to the fact that prices have increased
more in urban areas during the 2008/09 food price crisis. Also in
the North, there was a doubling of the cost of basic needs. The sharp
increase in prices also mean that households are devoting significantly
larger shares to food. 

Table \ref{tab:Utility-Consistent-Poverty} reports poverty measures
based on the regionally differentiated poverty lines using the FGT
index \citep{RePEc:ecm:emetrp:v:52:y:1984:i:3:p:761-66}. We have
aggregated the measures at the national, the rural/urban and the regional
level for all surveys. We have added the district level for the UNHS
surveys. Unfortunately, the UNPS 2010/11 is only representative up
to the regional level. The figures presented in 

At the national level, we find that poverty stood at about 38 percent
in 2005/06, about 7 percentage points higher than the official figure.
Over time, poverty has decreased, but slower than what official poverty
figures suggest. For instance, between 2005/06 and 2009/10, official
headcount poverty fell by about 23 percent. Our estimates suggest
the reduction over that period was only 5 percent. But in 2010/11,
a substantial poverty reduction is registered, bringing the reduction
over this larger time span closer to the official estimates. It is
easy to see why we find this pattern. A cost of basic needs approach
rooted in regional caloric requirements and differentiated consumption
bundles would be expected to shift poverty lines upward if prices
of commodities increase. Our poverty measure is likely to better reflect
the impact of the food price crisis than the national poverty measure
based on (consumer price index adjusted) poverty lines. This explanation
is supported by the fact that poverty increases for the urban poor
(from 13 to 16.3 percent) between 2005/06 and 2009/10, and the poorest
of the poor (the North Eastern province), two groups that are likely
to be especially hit by food price surges \citep{New6}.

If we aggregate by region, we find that, in 2005/06, poverty is lowest
in the Northern region, a region that has always been regarded as
the poorest in Uganda (see table \ref{tab:Official-poverty-headcount00s}).
However, these figures mask significant heterogeneity in the North.
Despite the fact that the North as a region is not ranked as the poorest
region, we find nevertheless that North East is consistently ranked
as the poorest province. The North East is also known as the Karamajong
region, a dry and conflict prone area inhabited by a nomadic people
that rely on livestock herding as their main livelihood activity.
It is well known that these people are among the most marginalized
groups in Uganda and vulnerable to frequent food and water shortages.
In addition, the Northern region is also the only one where poverty
seems to be on the rise over time. This fact corresponds to the feeling
that the North has been neglected, or at best, that the North has
not benefited much from government policies to eradicate growth. A
last contributing factor to the pattern may be due to sampling bias.
Conflicts in the North meant that, often, substantial parts of the
area were not included in the survey. Since conflict affected area
are likely to be poorer, leaving them out bias poverty downwards.
When stability returned around 2006, subsequent surveys added areas
that were previously affected by conflict, sharply increasing headcount
poverty. 

We also find that the Western region was the poorest in 2005/06 and
in 2010/11. It was briefly surpassed by the Eastern region during
the food price crisis. Indeed, as we mentioned before, the price of
matooke did not increase as much as the price of other commodities,
due to its localized nature. Our results are consistent with the analysis
of \citet{New1} and \citet{RePEc:oup:jafrec:v:12:y:2003:i:4:p:598-624}.
They both argue that a single consumption basket can not represent
the varied diets Ugandans in different parts of the country enjoy.
They therefore construct poverty lines that are different by spatial
domains, much as what we have been doing here. Their conclusion is
in line with our conclusion. Due to the fact that Northerners consume
much more calorie efficient staple foods (such as cassava and sorghum),
their cost of basic needs is lower, leading to a lower poverty line
than the national one. In contrast, in the Western region, people
tend to rely much more on matooke, which is a very expensive source
of calories. They both conclude that, when using region specific cost
of basic needs poverty lines, the Western Region was the poorest in
the 1990s.

\begin{table}
\caption{Utility Consistent Poverty Measures\label{tab:Utility-Consistent-Poverty}}


\centering{}%
\begin{tabular}{rccccccccc}
 & \multicolumn{3}{c}{2005/06} & \multicolumn{3}{c}{2009/10} & \multicolumn{3}{c}{2010/11}\tabularnewline
\cline{2-10} 
 & P0 & P1 & P2 & P0 & P1 & P2 & P0 & P1 & P2\tabularnewline
\cline{2-10} 
 &  &  &  &  &  &  &  &  & \tabularnewline
\hline 
National & 37.80 & 10.96 & 4.51 & 36.03 & 10.67 & 4.49 & 31.36 & 10.36 & 4.88\tabularnewline
 &  &  &  &  &  &  &  &  & \tabularnewline
Urban & 12.93 & 3.15 & 1.17 & 16.35 & 4.34 & 1.74 & 9.94 & 2.66 & 1.32\tabularnewline
Rural & 42.48 & 12.43 & 5.14 & 39.03 & 11.63 & 4.91 & 35.25 & 11.73 & 5.52\tabularnewline
 &  &  &  &  &  &  &  &  & \tabularnewline
Central & 36.70 & 11.44 & 4.92 & 33.54 & 9.80 & 4.09 & 21.70 & 8.48 & 5.03\tabularnewline
Eastern & 44.29 & 11.85 & 4.60 & 40.22 & 11.48 & 4.59 & 32.46 & 9.39 & 3.96\tabularnewline
Northern & 22.17 & 5.11 & 1.71 & 33.07 & 10.40 & 4.69 & 31.25 & 9.91 & 4.26\tabularnewline
Western & 44.57 & 13.95 & 6.06 & 35.29 & 10.65 & 4.58 & 41.14 & 13.89 & 6.26\tabularnewline
 &  &  &  &  &  &  &  &  & \tabularnewline
Kampala & 7.98 & 2.02 & 0.86 & 17.91 & 4.63 & 1.48 & 6.37 & 1.14 & 0.67\tabularnewline
Central 1 & 38.87 & 12.33 & 5.28 & 30.23 & 9.47 & 4.12 &  &  & \tabularnewline
Central 2 & 48.19 & 15.03 & 6.50 & 43.55 & 12.32 & 5.16 &  &  & \tabularnewline
East Central & 40.32 & 10.16 & 3.83 & 36.08 & 10.26 & 4.12 &  &  & \tabularnewline
Eastern & 45.92 & 12.95 & 5.13 & 43.60 & 12.47 & 4.98 &  &  & \tabularnewline
Mid Northern & 22.42 & 4.83 & 1.50 & 26.66 & 8.21 & 3.77 &  &  & \tabularnewline
North East & 55.96 & 16.36 & 6.53 & 70.56 & 28.10 & 13.84 &  &  & \tabularnewline
West Nile & 14.50 & 2.99 & 1.02 & 24.60 & 5.17 & 1.66 &  &  & \tabularnewline
Mid Western & 42.57 & 13.21 & 5.85 & 37.79 & 12.21 & 5.59 &  &  & \tabularnewline
South Western & 45.83 & 14.42 & 6.20 & 32.64 & 9.00 & 3.51 &  &  & \tabularnewline
\hline 
\end{tabular}
\end{table}


We already made the case above that poverty measures based on region
specific and utility consistent poverty lines are to be preferred
to poverty lines based on a single consumption basket from a theoretical
point of view. In addition, there is widespread skepticism amongst
researchers and observers about the official poverty figures disclosed
by the government. Our estimates are closer to what qualitative data
suggests, and are also in line with studies by independent researchers,
such as \citet{New2} and \citet{RePEc:oup:jafrec:v:12:y:2003:i:4:p:598-624}.


\section{Conclusion}

In this paper, we re-assess the evolution of poverty over the past
10 years in Uganda. Official figures suggest substantial poverty reduction,
but independent researchers note that the benefits of economic growth
have been shared unequally. In addition, casual observation does not
correspond to the rosy picture that official figures suggest. Other
indicators that define well being in a broader way, such as adult
literacy and maternal health, also put Uganda at a much lower level
than what would correspond to the disseminated poverty levels. 

One possible explanation for this divergence lies in the poverty line.
The poverty line that is currently in use to estimate official poverty
in Uganda has been constructed over a decade ago, using data from
a 1993/1994 survey. In addition, this poverty line relies on a single
food consumption basket for Uganda, despite the fact that Uganda consists
of a diverse set of regions, each with their own diets. These diets
are also exceptional in their difference in cost to obtain a certain
level of kilo-calories (or utility of that matter). Lumping all regions
together and assuming they require the same amounts of each commodity
disregards the cultural and agro-climatic diversity that typifies
Uganda.

We therefore follow \citet{RePEc:ucp:ecdecc:v:58:y:2010:i:3:p:449-474},
who propose an information-theoretic approach to constructing utility
consistent poverty lines. The idea is to construct different poverty
lines by spatial (or temporal) domain that yield a minimal amount
of kilo-calories given the demographic make-up of the region. These
poverty lines are then tested to check if they obey revealed preference
conditions. In particular, we check if the food baskets chosen in
all other regions are less expensive than the food basket chosen in
a particular region. If not, the individual could have chosen a cheaper
basket that yields the same utility. This violates the revealed preference
condition. We apply an information-theoretic approach that adjusts
consumption shares such that this revealed preference condition is
satisfied, while keeping the original diets in tact as much as possible.

Applying the above to three nationally representative surveys for
Uganda, we find that national poverty has been and remains higher
than the official figures suggest. Also somewhat controversial, we
find that not the North, but the West is the region that is poorest.
This can be re-conciliated by the fact that households living in the
West are disproportionally dependent on matooke for their caloric
requirements, while the North prefers (or has to rely on because of
climatic conditions) cassava and sorghum. Matooke is at least twice
as expensive as cassava, and three times as expensive as sorghum to
obtain a given amount of kilo-calories. This results in a much lower
poverty line for the North than the one for the West. Still, digging
deeper, we find that the figure for the North still veils quite some
differences. The karamajong region in the North remains the poorest
of all districts. 

\newpage{}\bibliographystyle{econometrica}
\bibliography{gap_paper}

\end{document}
